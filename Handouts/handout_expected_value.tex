\documentclass[11pt]{article}
\usepackage[utf8]{inputenc}
\usepackage{amsmath}
\usepackage{amssymb}
\usepackage{graphicx}
\usepackage{hyperref}
\usepackage[parfill]{parskip}
\let\oldemptyset\emptyset
\let\emptyset\varnothing


\title{\textbf{Esssentials of Applied Data Analysis\\
				IPSA-USP Summer School 2017}\newline\\
				Expected Value}

\author{Leonardo Sangali Barone\\ \href{leonardo.barone@usp.br}{leonardo.barone@usp.br}}
\date{jan/17}

\begin{document}

\maketitle

\section*{Expectation and mean value of a discrete random variable}


	\subsection*{Expectation and a simple game}

	Imagine a game where you toss a coin and get \$1 if the result is head and 0 if the result is tail. How much should you expect to earn?\\
		
	The expected value of a discrete random variable can be easily obtained by summing each result multiplied by probability of that result ocurring.\\
	
	\[E[X] = 0.5 * 0 + 0.5 * 1 = 0.5\]

	Now, imagine a game where you roll a dice and you can get \$1 times the number you get on the dice. How much should you expect to earn?\\

\[E[X] = \frac{1}{6} * 1 + \frac{1}{6} * 2 + \frac{1}{6} * 3 + \frac{1}{6} * 4 + \frac{1}{6} * 5 + \frac{1}{6} * 6 + \frac{1}{6} * 1 = 3.666\]


	\subsection*{Expectation and mean value of a discrete random variable}

	In more general term, the expectation or mean of a discrete random variable is:
	\[E[X] = \sum\limits_{i=1}^n x_i * P(X = x_i) = \sum\limits_{i=1}^n x_i * f(x_i)\]
where $x_i$ is an occurence of the variable X and $f(x_i)$ is the probability mass function (the probability that $x_i$ will occur).\\

	Note that, since the set of all $x_i$ is the set of all possible values for $X$, then
	\[E[X] = \sum\limits_{i=1}^n P(X = x_i) = \sum\limits_{i=1}^n f(x_i) = 1\]


	\subsection*{Expectation and mean value of a discrete random variable}
	Example: (Made-up) survey with 2000 respondents in 2014 Brazilian presidential elections.
	\newline\\
		\begin{tabular}{|c|c|c|}
\hline
	Candidate $(x)$ & \# of respondents & $P(X=x)$\\
	\cline{1-3}
	Dilma & 800  & 0.40\\
	Aécio &  500 & 0.25\\
	Marina & 400 & 0.20\\
	Other/Null/DK & 500 & 0.25 \\
	\cline{1-3}
	Total & 2000 & 1.00\\
\hline
\end{tabular}
	\newline\\
	Can we calculate $E[X]$?


	\subsection*{Expectation and mean value of a discrete random variable}
	When all $P(X=x_i)$ is the same for every $x_i$, we can simplify the expression of $E[X]$ to:
	\[E[X] = \frac{1}{n}\sum\limits_{i=1}^n x_i = \frac{x_1 + x_2 + ... + x_n}{n}\]
	which is what we normally do to calculate avareges in daily life.

	\subsection*{Expectation and mean value of a discrete random variable}
	Some properties of the mean:
	\[E[a*X] = a * E[X]\]
	\[E[X+ b] = E[X] + b\]
	So what? Well, if you multiply a variable by a number ($a$) to generate a new variable, the mean of the new variable is the mean of the old variable times $a$.
	\newline\\Also, if you sum a quantity $b$ to a random variable, the mean of the result variable will be the mean of the original variable plus $b$.
	\newline\\Let's try it later on an statistical software!

	\subsection*{Expectation and variance of a discrete random variable}
	Another important quantity of a random variable is the variance. The name is self-explanatory: the variance measures how spread-out a variable is.

	\subsection*{Expectation and variance of a discrete random variable}
	The variance of a random variable is also an expectation:
		\[Var[X] = \sum\limits_{i=1}^n[x_i - E[X]]^2 * P(X=x_i) =\sum\limits_{i=1}^n[x_i - E[X]]^2 * f(x_i)\]
where $x_i$ is an occurence of the variable X, E[X] is the expected value of X and $f(x_i)$ is the probability mass function (the probability that $x_i$ will occur).

	\subsection*{Expectation and variance of a discrete random variable}
	Coin game (where heads pays off \$1 and tails pays off \$0):
	\[Var[X] = \sum\limits_{i=1}^n[x_i - E[X]]^2 * f(x_i) = [0.5 - 0.5]^2 * 0 + 0.5 * 1 = 0.5\]


	\subsection*{Expectation and variance of a discrete random variable}
	Some properties of the variance:
	\[Var[a*X] = a^2 * E[X]\]
	\[Var[X+ b] = Var[X]\]
	So what? Well, if you multiply a variable by a number ($a$) to generate a new variable, the variance of the new variable is the variance of the old variable times $a^2$.
	\newline\\Also, if you sum a quantity $b$ to a random variable, the variance of the result variable will equal to the variance of the original variable.
	\newline\\Let's try it later on an statistical software!

	\subsection*{Expectation and variance of a discrete random variable}
	Some properties of the variance:
	\[Var[a*X] = a^2 * E[X]\]
	\[Var[X+ b] = Var[X]\]
	So what? Well, if you multiply a variable by a number ($a$) to generate a new variable, the variance of the new variable is the variance of the old variable times $a^2$.
	\newline\\Also, if you sum a quantity $b$ to a random variable, the variance of the result variable will equal to the variance of the original variable.
	\newline\\Let's try it later on an statistical software!

	\subsection*{Expectation, mean, variance and standard deviation}
	Notation:
	\[E[X] = \mu[X] = \mu\]
	\[Var[X] = \sigma^2[X] = \sigma^2\]
	The standard deviation ($\sigma$) of a variable is
	\[\sigma = \sqrt{Var[X]} = \sqrt{\sigma^2}\]
	Another way to calculate the variance is simply doing: 
	\[Var[X] = E[X^2] - (E[X])^2\]	





\end{document}
