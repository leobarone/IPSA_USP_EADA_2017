\documentclass[11pt]{article}
\usepackage[utf8]{inputenc}
\usepackage{amsmath}
\usepackage{amssymb}
\usepackage{graphicx}
\usepackage{hyperref}
\usepackage[parfill]{parskip}
\let\oldemptyset\emptyset
\let\emptyset\varnothing


\title{\textbf{Esssentials of Applied Data Analysis\\
				IPSA-USP Summer School 2017}\newline\\
				Joint Distributions}

\author{Leonardo Sangali Barone\\ 	\href{leonardo.barone@usp.br}{leonardo.barone@usp.br}}
\date{jan/17}

\begin{document}

\maketitle


\section*{Joint Distribution}

	Causal theories of politics necessarily involve expected relationships among concepts or variables. As such, we want to study joint distributions; marginal distributions are a natural extension as we will see in a moment.
	\newline\\
	For discrete variables, we can use contingency tables to represent the joint frequency distribution for two random variables.


	\subsection*{Joint Distribution - legislators}
	
	Example: two variables, gender ($Y$) and political party ($X$)\\
	If we take the relative frequence of the cells we get:\\

	\begin{tabular}{|c|ccc|}
\hline
	Y/X & Party A ($A$) & Party B ($B$) & Party C ($C$)\\
\hline
	Women ($W$) & $P(W \cap A)$ & $P(W \cap B)$ & $P(W \cap C)$\\
	Man ($M$) & $P(M \cap A)$ & $P(M \cap B)$ & $P(M \cap C)$\\
\hline
\end{tabular}\newline\\
Note: joint distributions are represented, guess what, by joint probabilities!


	\subsection*{Joint Distribution }
	We can write the joint probabilities as \[P(Women \cap \text{Party B}) \text{ or }P(W \cup B)\] or, if we have named the variables \[P(Gender = Women, Party = \text{Party B})\] \[P(X = W, Y = B)\] These notations are equivalent if everything is well named.



	\subsection*{Joint Distribution - dice}
	Joint distributions can be build from the process that generate the data (dice) or from a sample.
	\newline\\
	Example: roll a dice. Prime \emph{vs} not prime (Y); and even \emph{vs} odd (X). If we take the relative frequencies we get:\newline\\
	\begin{tabular}{|c|cc|}
\hline
	Y/X & Even ($E$) & Odd ($O$)\\
\hline
	Prime ($I$) & $P(I \cap E) = 1/6$ & $P(I \cap O) = 2/6$\\
	Not Prime ($N$) & $P(N \cap E) = 2/6$ & $P(N \cap O)= 1/6$ \\
\hline
\end{tabular}\newline\\


	\subsection*{Joint Distribution - sex and political affiliation in Fakeland}

	We could do the same using our Fakeland example (let's not use numbers here).

	Example:\newline\\
	\begin{tabular}{|c|ccc|}
\hline
	Sex/Party & Conservative ($C$) & Independent ($I$) & Socialist ($S$)\\
\hline
	Women ($W$) & $P(W \cap C)$ & $P(W \cap I)$ & $P(W \cap S)$\\
	Man ($M$) & $P(M \cap C)$ & $P(M \cap I)$ & $P(M \cap S)$ \\
\hline

\end{tabular}\newline\\

Now, use the fake dataset to build the table above. Remember that what you will build is the joint \emph{sample} distribution (not the ``True'' Fakeland distribution).

	\subsection*{Joint Distribution - conditional probability notation}
	
	All of the notations below are equivalent.
\[P(X=x_i|Y=1) = P(X|Y=1) = \frac{P(X=x, Y =1)}{P(Y=1)} = \frac{P(X \cap (Y =1))}{P(Y=1)} \]


	\subsection*{Joint Distribution - Marginal Probabilities}
	The marginal probability of an event A is the probability that A will occur unconditional on all the other events on which A may depend. It is very easy to comprehend that in our example. If we take the relative frequencies we get:
	\newline\\
	Example:\newline\\
	\begin{tabular}{|c|ccc|c|}
\hline
	Sex/Party & Conservative ($C$) & Independent ($I$) & Socialist ($S$) & Marginal\\
\hline
	Women ($W$) & $P(W \cap C)$ & $P(W \cap I)$ & $P(W \cap S)$ & P(W)\\
	Man ($M$) & $P(M \cap C)$ & $P(M \cap I)$ & $P(M \cap S)$ & P(M) \\
\hline
	Marginal & $P(C)$ & $P(I)$ & $P(S)$ & 1\\
\hline

\end{tabular}\newline\\



	\subsection*{Joint Distribution - Marginal Probabilities}
	
	We can calculate the Marginal Probability by simplying summing the probability of $A$ hapenning conditional on all other events on which $A$ depend (partitions o $B$):
	
\[P(A) = P(A\cap B_1) + P(A\cap B_2) + ... + P(A \cap B_n) = \sum\limits_{i=1}^n P(A\cap B_i)  \]
or
\[P(A) = P(B_1)*P(A|B_1) + P(B_2)*P(A|B_2) +...+ P(B_n)*P(A|B_n)= \]
\[= \sum\limits_{i=1}^n P(B_i)*P(A|B_i) \]

This means that one averages over other events and focuses on the one event, A, of interest.

	\subsection*{Joint Distribution - Independence - Cards}
	What does happen with the joint distribution of two random variables if they are independent of each other?
	
	Example: choose a card from a deck. Calculate the relative frequencies:\newline\\
	\begin{tabular}{|c|cccc|c|}
\hline
	Y/X & Hearts & Spades & Clubs & Diamonds & Marginal \\
\hline
	King & 1/52 & 1/52 & 1/52 & 1/52 & 4/52\\
	Queen & 1/52 & 1/52 & 1/52 & 1/52 & 4/52\\
	Other & 11/52 & 11/52 & 11/52 & 11/52 & 44/52\\
\hline
	Marginal & 13/52 & 13/52 & 13/52 & 13/52 & 52/52\\
\hline

\end{tabular}\\

We got a king. What is the probability that it is the king of hearts? \[P(H|K)=1/4\]
We got a queen. What is the probability that it is the queen of hearts? \[P(H|Q)=1/4\]	
We got any other card. What is the probability that it is a card of hearts? \[P(H|O)=1/4\]

If the marginals probabilities are equal to the conditional probabilities, than the two variables are independent from each other.
\[P(H) = P(H|K)=P(H|Q)=P(H|O) = 1/4\]
Under independence:
\[P(H\cap K) = P(H)*P(K) = 1/4*1/13 = 1/52\] or
\[P(Y= Hearts, X = King) = P(Y = Hearts)*P(X = King)\]

	\subsection*{Joint Distribution - sex and political affiliation in Fakeland}

	Let's go back to our empirical example. If you completed the sample joint distribution, you got the table below:

Count: 

\begin{tabular}{|c|ccc|c|}
\hline
	Sex/Party & Conservative ($C$) & Independent ($I$) & Socialist ($S$) & Marginal\\
\hline
	Women ($W$) & 3 & 8 & 4 & 15\\
	Man ($M$) & 3 & 7 & 5 & 15 \\
\hline
	Marginal & 6 & 15 & 9 & 30\\
\hline
\end{tabular}

Proportion relative to Total: 

\begin{tabular}{|c|ccc|c|}
\hline
	Sex/Party & Conservative ($C$) & Independent ($I$) & Socialist ($S$) & Marginal\\
\hline
	Women ($W$) & 0.10 & 0.27 & 0.13 & 0.50\\
	Man ($M$) & 0.10 & 0.23 & 0.17 & 0.50 \\
\hline
	Marginal & 0.20 & 0.50 & 0.30 & 1.00\\
\hline
\end{tabular}

Proportion relative to Rows: 

\begin{tabular}{|c|ccc|c|}
\hline
	Sex/Party & Conservative ($C$) & Independent ($I$) & Socialist ($S$) & Marginal\\
\hline
	Women ($W$) & 0.20 & 0.54 & 0.26 & 1.00\\
	Man ($M$) & 0.20 & 0.47 & 0.33 & 1.00 \\
\hline
	Marginal & 0.20 & 0.50 & 0.30 & 1.00\\
\hline
\end{tabular}

Proportion relative to Columns: 

\begin{tabular}{|c|ccc|c|}
\hline
	Sex/Party & Conservative ($C$) & Independent ($I$) & Socialist ($S$) & Marginal\\
\hline
	Women ($W$) & 0.50 & 0.53 & 0.13 & 0.45\\
	Man ($M$) & 0.50 & 0.47 & 0.17 & 0.55 \\
\hline
	Marginal & 1.00 & 1.00 & 1.00 & 1.00\\
\hline
\end{tabular}\\

Can you tell if the two variables are independent of each other just by looking at the tables? We are going to train this a lot with real data using Stata.

\end{document}
