\documentclass[11pt]{article}
\usepackage[utf8]{inputenc}
\usepackage{amsmath}
\usepackage{amssymb}
\usepackage{graphicx}
\usepackage{indentfirst}
\usepackage{hyperref}
\let\oldemptyset\emptyset
\let\emptyset\varnothing


\title{\textbf{Esssentials of Applied Data Analysis\\
				IPSA-USP Summer School 2017}\newline\\
				Handout - Discrete Random Variables}

\author{Leonardo Sangali Barone\\ \href{leonardo.barone@usp.br}{leonardo.barone@usp.br}}
\date{jan/17}

\begin{document}

\maketitle

\section*{Discrete Random Variables}

	\subsection*{Dice}
	
	Let's call $X$ the variable that indicates the result we obtain when we roll a 6-side dice. This variable can assume 6 different values, \{1,2,3,4,5,6\}, and all of them have the same probability, $P(X=x_i) = 1/6$.\newline\\
	\begin{tabular}{|c|c|}
\hline
	$x_i$ & $P(X=x_i)$\\
	\cline{1-2}
	1 & 1/6\\
	2 & 1/6\\
	3 & 1/6\\
	4 & 1/6\\
	5 & 1/6\\
	6 & 1/6\\

\hline
\end{tabular}\\

Note: The realization of a random variable is a particular value that it takes and we usually represent it with a small letter (e.g. $x_i$ is a realization of $X$)


	\subsection*{Discrete Random variables - dice}
	Let's call $Y$ the variable that indicates the result we obtain when we roll a pair of dice and sum the results. What is the distribution of $Y$?\\
	
	\begin{tabular}{|c|c|c|c|}
\hline
	$y_i$ & $P(Y=y_i)$ & $y_i$ & $P(Y=y_i)$\\
	\cline{1-4}
	1 & 0 & 7 & 6/36\\
	2 & 1/36 & 8 & 5/36\\
	3 & 2/36 & 9 & 4/36\\
	4 & 3/36 & 10 & 3/36\\
	5 & 4/36 & 11 & 2/36\\
	6 & 5/36 & 12 & 1/36\\
\hline
\end{tabular}\\

	Let's call $X$ the variable that indicates the number of heads we obtain when we toss a coin 5 times. This variable can assume 6 different values: \{0,1,2,3,4,5\}.\\
	
	\begin{tabular}{|c|c|}
\hline
	$x_i$ & $P(X=x_i)$\\
	\cline{1-2}
	0 & $(0.5)^5$\\
	1 & $(0.5)^1$\\
	2 & $(0.5)^2$\\
	3 & $(0.5)^3$\\
	4 & $(0.5)^4$\\
	5 & $(0.5)^5$\\
\hline
\end{tabular}

	\subsection*{Example -- parties and cabinets}
	Hypothetically, we could build distribution tables like we just did for the dice examples for a lot of discrete variables in political science.\\
	
	Example: Number of cabinets controlled by Party A in Dilma's second term ($X$).\\
	
	\begin{tabular}{|c|c|}
\hline
	$x_i$ & $P(X=x_i)$\\
	\cline{1-2}
	none & $P(X=0)$\\
	1 & $P(X=1)$\\
	2 & $P(X=2)$\\
	3 or more & $P(X \geq 3)$\\
\hline
\end{tabular}\\

Note: Why can we fill the table with numbers when we talk about dices and coins and we can't it with number of cabinets?\\


	\subsection*{Example -- Policy options}
	Example: Household policy options for federal government ($X$).\\
	
	\begin{tabular}{|c|c|}
\hline
	Policy option ($x_i$) & $P(X=x_i)$\\
	\cline{1-2}
	Credit to individuals/families (A) & $P(X=A)$\\
	Transfers to municipal programs (B) & $P(X=B)$\\
	Partneship with urban settlement movements (C) & $P(X=C)$\\
\hline
\end{tabular}\\

	\subsection*{Presidential election on TV}
	
	Example: During 2014 Brazilian presidental elections, a voter turned her TV on during HEG (Horário Eleitoral Gratuito). What is the probability, for each candidate, that she saw them (or their adds) first?\\
	
	\begin{tabular}{|c|c|c|}
\hline
	Candidate $(x_i)$ & Time on HEG & $P(X=x_i)$\\
	\cline{1-3}
	Dilma & 11 minutes and 24 seconds  & 0.380\\
	Aécio & 4 minutes and 35 seconds & 0.153\\
	Marina & 2 minutes and 03 seconds & 0.068\\
	Other & 11 minutes and 58 seconds & 0.399\\
	\cline{1-3}
	Total & 30 minutes and 00 seconds & 1.000\\
\hline
\end{tabular}\\

Note: we haven't measured the realization of the variable, but we know the data generation process - and we are not talking about dices and coins!

	\subsection*{Discrete Random variables - drawing graphs}
	
	We can represent the distribution of a random discrete variable with both a table, as we have done, or a histogram. We are going to do it in our laboratory, but can you graph the distributions we have just seen?\\

	It is very easy to convert a frequency table into a a probability distribution for discrete random variables. The probability that a specific case drawn at random from a sample has a specific value, $x_i$, is the relative frequency of the value $x_i$.\\

	$Y$ the variable that indicates the result we obtain when we roll a pair of dice and sum the results.\newline\\
	\begin{tabular}{|c|c|c|c|}
\hline
	$y_i$ & $P(Y=y_i)$ & $y_i$ & $P(Y=y_i)$\\
	\cline{1-4}
	1 & 0 & 7 & 6/36\\
	2 & 1/36 & 8 & 5/36\\
	3 & 2/36 & 9 & 4/36\\
	4 & 3/36 & 10 & 3/36\\
	5 & 4/36 & 11 & 2/36\\
	6 & 5/36 & 12 & 1/36\\
\hline
\end{tabular}
\newline\\
Can you make the graph of this probability distribution?


	\subsection*{Sample distribution}
	We differentiate probability distribution from a sample distribution. We can build a sample distribiution from a frequency table (ocurrences of the variable in the sample!). More generally, we can build distribuitions from the realizations of the variable (e.g. number of seats in a Legislative, Moore and Siegel, ch 10, pp. 202-203).

	\subsection*{Sample distribution}
	Example: (Made-up) survey with 2000 respondents in 2014 Brazilian presidential elections.
	\newline\\
		\begin{tabular}{|c|c|c|}
\hline
	Candidate $(x_i)$ & \# of respondents & $P(X=x_i)$\\
	\cline{1-3}
	Dilma & 800  & 0.40\\
	Aécio &  500 & 0.25\\
	Marina & 400 & 0.20\\
	Other/Null/DK & 500 & 0.25 \\
	\cline{1-3}
	Total & 2000 & 1.00\\
\hline
\end{tabular}
	   
	\subsection*{Discrete Random variables - Cumulative Distribution}
	For ordered or integer discrete random variables ($X$), we might be interested in $P(X \leq x_i)$ instead of $P(X=x_i)$, which is just the probability function $f(x_i)$. We can also think of $P(X \leq x_i)$ as a function, $F(X)$, and it is defined as:
	\[F(x_i) = P(X \leq x_i) =  \sum\limits_{i\leq x_i} f(i)\]
	This function is called the cumulative distribution function.

	\subsection*{Discrete Random variables - drawing graphs}
	$Y$ the variable that indicates the result we obtain when we roll a pair of dice and sum the results.\newline\\
	\begin{tabular}{|c|c|c|c|}
\hline
	$y_i$ & $P(Y=y_i)$ & $y_i$ & $P(Y=y_i)$\\
	\cline{1-4}
	1 & 0 & 7 & 6/36\\
	2 & 1/36 & 8 & 5/36\\
	3 & 2/36 & 9 & 4/36\\
	4 & 3/36 & 10 & 3/36\\
	5 & 4/36 & 11 & 2/36\\
	6 & 5/36 & 12 & 1/36\\
\hline
\end{tabular}
\newline\\
Can you make the graph of this cumulative distribution?

\end{document}
